%%%%%%%%%%%%%%%%%%%%%%%%%%%%%%%%%%%%%%%%%%%%%%
% Head matter - can we try to be consistent on
% included packages
\documentclass[xcolor=table	]{beamer}
\mode<presentation>
{\usetheme{default}
 \usecolortheme{default}
 \usefonttheme{default}
 \setbeamertemplate{navigation symbols}{}
 \setbeamertemplate{caption}[numbered]} 

\usepackage{syntax}
\usetheme[secheader]{Boadilla}

% Add Better I for interpretation 
\usepackage{mathrsfs}
\usepackage{mathtools}
\DeclarePairedDelimiter\ev{\langle}{\rangle}%


\usepackage{array}
%\newtheorem{lemma}{Lemma}
\usepackage[customcolors]{hf-tikz}
\usepackage{tikz}

\newcommand{\ov}[1]{\mkern 1.5mu\overline{\mkern-1.5mu#1\mkern-1.5mu}\mkern 1.5mu}


\usepackage{xcolor}% http://ctan.org/pkg/xcol
\definecolor{OliveGreen}{rgb}{0.2,0.7,0.2}
\definecolor{airforceblue}{rgb}{0.36, 0.54, 0.66}
\definecolor{awesome}{rgb}{1.0, 0.13, 0.32}
\definecolor{brilliantrose}{rgb}{1.0, 0.33, 0.64}
\definecolor{charcoal}{rgb}{0.21, 0.27, 0.31}
\definecolor{britishracinggreen}{rgb}{0.0, 0.26, 0.15}
\definecolor{darkgreen}{rgb}{0.0, 0.2, 0.13}
\definecolor{darkpastelgreen}{rgb}{0.01, 0.75, 0.24}
\definecolor{lavenderindigo}{rgb}{0.58, 0.34, 0.92}
\definecolor{ultramarine}{RGB}{0,32,96}
\definecolor{mycolor}{rgb}{.5,.05,.05}
\definecolor{britishracinggreen}{rgb}{0.0, 0.26, 0.15}
\definecolor{burntorange}{rgb}{0.8, 0.33, 0.0}
\definecolor{coolblack}{rgb}{0.0, 0.18, 0.39}
\definecolor{harvestgold}{rgb}{0.85, 0.57, 0.0}
\definecolor{harvardcrimson}{rgb}{0.79, 0.0, 0.09}

\definecolor{wrongultramarine}{rgb}{0.07, 0.04, 0.56}

 \setbeamertemplate{footline}{%
   \raisebox{5pt}{\makebox[\paperwidth]{\hfill\makebox[10pt]{\scriptsize\insertframenumber}}}}
%%%%%%%%%%%%%%%%%%%%%%%%%%%%%%%%%%%%%%%%%%%%%%
% Formatting for title page
\title[QBF]{Autarkies for DQCNF}
\author{Oliver Kullmann \inst{1} \and \textbf{Ankit Shukla} \inst{2}}
\institute{\inst{1} Swansea university \inst{2} JKU, Austria}
%{\large International Workshop on QBFs and Beyond}
%\date{10 September 18}
%%%%%%%%%%%%%%%%%%%%%%%%%%%%%%%%%%%%%%%%%%%%%%

\begin{document}
\begin{frame}
  \titlepage
\end{frame}

\begin{frame}{TODO}
 Later then one defines "autarkies" in general,
 
 and E1,A1 etc.
 
 Plus the main lemma, 
 
 the SAT-equivalence, 
 
 and explaining "autarky systems" and 
 
 confluence.
\end{frame}

%\begin{frame}{DQCNF}
%Dependency Quantified Boolean Formulas (DQBF) extend QBF \\
%with \textbf{Henkin quantifiers}, which allow for non-linear dependencies \\ between the quantified variables. 
%\hfsetfillcolor{green!50}
%\hfsetbordercolor{green!50!black}
%\begin{equation}\label{e:barwq3}
%  \begin{split}
%	\tikzmarkin{c}(0.05,-0.6)(-0.05,0.65)F \, = \, \forall u1, u2 \, \exists e1(u1), e2(u1, u2), e3(u1, u2) . F_0\\
%	F_0 \, = \,(u1 \lor e1) \land (u2 \lor e2) \land (u1 \lor u2 \lor e3)
%	\tikzmarkend{c}
%\end{split}
%\end{equation}	
%\begin{equation}
%\begin{split}
% \overbrace{\tikz[baseline]{
%		\node[fill=yellow!30,anchor=base] (t1)
%		{$\forall x_1, x_2 \, \exists y_1(x_1), y_2(x_1, x_2)$};
%}}^{\displaystyle Dep. \, quantifier \, prefix} : \underbrace{\tikz[baseline]{
%		\node[fill=green!30,anchor=base] (t1)
%		{$(x_1 \lor y_1) \land (x_2 \lor y_2 \lor \neg y_1)$};
%}}^{\displaystyle CNF \, formula}
%\end{split}
%\end{equation}
%
%
%\begin{alertblock}{Dependency set}
% $y_1 \to \{x_1 \}$
%
% $y_2 \to  \{ x_1, x_2 \}$
%\end{alertblock}
%  
%\end{frame}

\begin{frame}{Solve DQBF: SAT or UNSAT?}

%\vspace{0.2cm}

\begin{alertblock}{1. DQCNF formula:}
	{
		$\forall x_1,x_2,x_3 \, \exists y_1(x_1,x_2) \, \exists y_2(x_2,x_3) \, \exists y_3(x_1) : \newline$
		$(y_{1} \vee x_1) \, \wedge \, (\ov{y_1} \vee x_2)  \newline$ 
		$ \, \wedge \, (\ov{y_2} \vee \ov{x_2} \vee x_3)  \newline \, \wedge \, (y_3 \vee \ov{x_1} \vee x_2) \, \wedge \, (\ov{y_3} \vee x_1)$
		\vspace{0.2cm}
	}
\end{alertblock}
\pause

Rules of the Game: \\

\begin{itemize}
	\pause
	
	\item Substitute functions in place of existential variables.
	\pause
	
	\item Clauses are satisfied.
\end{itemize}
\pause
 
\textbf{Ques.} What are allowed values? \\
\pause

Ans: Any function of dependency set variables.
\end{frame}

\begin{frame}{Solve DQBF: SAT or UNSAT?}

%\vspace{0.2cm}

\begin{alertblock}{1. DQCNF formula:}
	{
		$\forall x_1,x_2,x_3 \, \exists y_1(x_1,x_2) \, \exists y_2(x_2,x_3) \, \exists y_3(x_1) : \newline$
		$(y_{1} \vee x_1) \, \wedge \, (\ov{y_1} \vee x_2)  \newline$ 
		$ \, \wedge \, (\ov{y_2} \vee \ov{x_2} \vee x_3)  \newline \, \wedge \, (y_3 \vee \ov{x_1} \vee x_2) \, \wedge \, (\ov{y_3} \vee x_1)$
		\vspace{0.2cm}
	}
\end{alertblock}

\begin{block}{Selected Boolean functions}
 \only<2> {
$y_1 \to 0$, $ y_1 \to 1 \newline$
$y_2 \to 0$, $ y_2 \to 1 \newline$
$y_3 \to 0$, $ y_3 \to 1$
}
 \only<3> {
	$y_1 \to 0$, $ y_1 \to 1 \newline$
	$y_2 \to 0$, $ y_2 \to 1 \newline$
	$y_3 \to 0$, $ y_3 \to 1$, $y_3 \to x_1$, $ y_3 \to \ov{x_1}$ 
}

\only<4> {
	$y_1 \to 0$, $ y_1 \to 1$, $y_1 \to x_1$, $ y_1 \to \ov{x_1}$, $y_1 \to x_2$, $ y_3 \to \ov{x_2} \newline$
	$y_2 \to 0$, $ y_2 \to 1 \newline$
	$y_3 \to 0$, $ y_3 \to 1$, $y_3 \to x_1$, $ y_3 \to \ov{x_1}$ 
}

\only<5> {
	$y_1 \to 0$, $ y_1 \to 1$, $y_1 \to x_1$, $ y_1 \to \ov{x_1}$, $y_1 \to x_2$, $ y_1 \to \ov{x_2} \newline$
	$y_2 \to 0$, $ y_2 \to 1, y_2 \to x_2$, $ y_2 \to \ov{x_2}$, $y_2 \to x_3$, $ y_2 \to \ov{x_3} \newline$
	$y_3 \to 0$, $ y_3 \to 1$, $y_3 \to x_1$, $ y_3 \to \ov{x_1}$ 
}
\end{block}

\end{frame}


\begin{frame}{}
\begin{alertblock}{DQCNF formula:}
	{
		$\forall x_1,x_2,x_3 \, \exists y_1(x_1,x_2) \, \exists y_2(x_2,x_3) \, \exists y_3(x_1) : \newline$
		$(y_{1} \vee x_1) \, \wedge \, (\ov{y_1} \vee x_2)  \newline$ 
		$ \, \wedge \, (\ov{y_2} \vee \ov{x_2} \vee x_3)  \newline \, \wedge \, (y_3 \vee \ov{x_1} \vee x_2) \, \wedge \, (\ov{y_3} \vee x_1)$
		\vspace{0.2cm}
	}
\end{alertblock}
\begin{exampleblock}{Solve the DCNF}
	{   \only Three choices $y_1, y_2, y_3$ ...
		\pause 
		\begin{enumerate}
		%	\pause
		\only \item Pick $y_2$\\
		\pause 
		  $\overline{y_2}$ is pure, $y_2 \rightarrow 0$ \newline 
		\pause 
		$\forall x_1,x_2,x_3 \, \exists y_1(x_1,x_2) \, \exists y_3(x_1)  : (y_1 \vee x_1) \wedge (\overline{y_1} \vee x_2)
		\, \wedge \, (1 \vee \ov{x_2} \vee x_3)  \, \wedge \, (y_3 \vee \ov{x_1} \vee x_2) \, \wedge \, (\ov{y_3} \vee x_1) \newline$ \\ 
        \pause 
        $\forall x_1,x_2,x_3 \, \exists y_1(x_1,x_2) \, \exists y_3(x_1)  : (y_1 \vee x_1) \wedge (\overline{y_1} \vee x_2)
        \, \wedge \, (y_3 \vee \ov{x_1} \vee x_2) \, \wedge \, (\ov{y_3} \vee x_1)$ \\ 
	    (removing the third clause).
		%	Both these autarkies are also $Eaut_1$-autarkies.
		\end{enumerate}
	}
\end{exampleblock}

%\textbf{lean kernel} of the original formula.
\end{frame}


\begin{frame}{}
\begin{alertblock}{DQCNF formula:}
	{
		$\forall x_1,x_2,x_3 \, \exists y_1(x_1,x_2) \, \exists y_2(x_2,x_3) \, \exists y_3(x_1) : \newline$
		$(y_{1} \vee x_1) \, \wedge \, (\ov{y_1} \vee x_2)  \newline$ 
		$ \, \wedge \, (\ov{y_2} \vee \ov{x_2} \vee x_3)  \newline \, \wedge \, (y_3 \vee \ov{x_1} \vee x_2) \, \wedge \, (\ov{y_3} \vee x_1)$
		\vspace{0.2cm}
	}
\end{alertblock}
\begin{exampleblock}{Solve the DCNF}
	{ \begin{itemize}
		\item Pick $y_3$: $y_3 \rightarrow x_1$  \\
			\pause
			
			\pause 
			$\forall x_1,x_2,x_3 \, \exists y_1(x_1,x_2)  : (y_1 \vee x_1) \wedge (\overline{y_1} \vee x_2)
			\, \wedge \, (x_1 \vee \ov{x_1} \vee x_2) \, \wedge \, (\ov{x_1} \vee x_1) \newline$ \\ 
			\pause 
			$\forall x_1,x_2,x_3 \exists y_1(x_1,x_2) : (y_1 \vee x_1) \wedge (\overline{y_1} \vee x_2)$
			
			removing the fourth and fifth clauses.	
			%	Both these autarkies are also $Eaut_1$-autarkies.
		\end{itemize}
	}
\end{exampleblock}
\pause
\begin{alertblock}{Reduced DQCNF formula:}
	{
			$\forall x_1,x_2,x_3 \exists y_1(x_1,x_2) : (y_1 \vee x_1) \wedge (\overline{y_1} \vee x_2)$
	}
\end{alertblock}

%\textbf{A1 Autarky lean kernel} of the original formula.
\end{frame}


%\begin{frame}{Autarkies for DQBF}
%``Autarkies" are some form of redundancies in CNFs, \newline
%with good theoretical properties \newline
%
%We present Autarky-system to determine these redundancies for DQBF \newline 
%
%%A1, E1, A1 + E1 \newline
%
%Present complete translation using an example 
%
%\end{frame}

%\begin{frame}{Autarkies}
%
%Autarkies for SAT are partial assignments for boolean CNF, \newline 
% which either satisfy a clause or leave it untouched. \newline
% 
%$F$ =  $(a) \land (a \lor b)$ \newline
% 
%Nontrivial Autarky:  $<b \to 1>$ \newline 
%
%Reduced formula: \newline 
%$F^{\prime}$ =  $(a)$
% 
%\end{frame}

\begin{frame}{Autarkies}
An autarky for a clause-set $F$ is a \textbf{partial assignment} $\varphi$ iff for every clause $C \in F$ either $\varphi$ does not ``touch" $C$, i.e., $var(\varphi) \cap var(C) = \emptyset$, \newline or $\varphi$ satisfies $C$, i.e., $\varphi \ast \{C\} = \top$.\newline

%
%%($var(\phi) \, \cap \, var(C)$ $\neq$ $\phi$)
%An autarky $\phi$ for $F$ is \textbf{trivial} if $var(\phi) \, \cap \, var(C)$ = $\phi$ \newline

\end{frame}

\begin{frame}{}
The extreme cases:
\begin{enumerate}
	\item The empty partial assignment $\ev{}$ is an autarky for every F.
	
	\item A satisfying assignment for F is also an autarky for F. \newline
\end{enumerate}

A \textbf{lean} clause-set is a clause-set $F$ which has
only the trivial autarky. \newline 

The class of all lean clause-sets is $lean$.

\end{frame}


\begin{frame}{Autarkies}

Applications: 
\begin{enumerate}
    \item Study of Unsatisfiability
    \item Pre-processing 
    \item Inprocessing
\end{enumerate}

 \begin{exampleblock}{Challenge}
 Finding an autarky for DQCNF is as hard as finding a satisfying assignment.
\end{exampleblock}
    
  \begin{alertblock}{Our solution: Autarky Systems}
  \begin{itemize}
      \item restricting the range of autarkies to a more feasible domain
      \item maintain the good general properties of arbitrary autarkies
  \end{itemize}
 
  \end{alertblock}

\end{frame}

\begin{frame}{Basic Lemmas}
 \begin{lemma}[1]
   $\varphi$ $\ast$ $F$ is satisfiability-equivalent to $F$ for an autarky $\varphi$ of $F$.
 \end{lemma}
 \begin{proof}
 	($\leftarrow$) A satisfying assignment of F satisfies also
 	$\varphi \ast F$, since just clauses have been removed. 
 	
 	 ($\to$) If $\phi$ is a total satisfying assignment for $\varphi \ast F$, then $\varphi \cup \phi$ is a (partial) satisfying assignment for F.
 \end{proof}
\end{frame}

\begin{frame}{Basic Lemmas}
  \begin{lemma}[2]
  The composition of two autarkies is again an autarky.
 \end{lemma}
 
  \begin{lemma}[3]
   Given a DQCNF $F$, the largest lean sub-DQCNF $N_{a} (F)$ is also \newline obtained by applying autarky-reduction to $F$ as long as possible.
 \end{lemma}
 
\end{frame}


%\begin{frame}{Autarky systems}
%
%Allow restricted notions of autarkies. \newline
%\begin{itemize}
%    \item For a DQCNF $F$ we write $Auk(F)$ for the set of all autarkies.  
%    \item An autarky system $A$ allows to consider subsets $A(F) \subseteq Auk(F)$.
%\end{itemize}
%
%\vspace{0.5cm}
%
%Required conditions for an autarky system:
%\begin{enumerate}
%    \item DQCNF $F$ and $\phi$, $\psi \in A(F)$ it
%must always hold $\phi \circ \psi \in A(F)$ (closure under composition).
%
%    \item DQCNF $F \subseteq F_0$ it must always hold $A(F_0) \subseteq A(F)$ (removal of clauses does not remove A-autarkies).
%\end{enumerate}
%
%A-satisfiability means satisfiability by a series of A-
%autarkies. \newline
%
%A-leanness means there there are no nontrivial A-autarkies.
%\end{frame}

%\begin{frame}{Additional concepts}
%
%Fundamental conditions on the Autarky system:
%
%\begin{enumerate}
%    \item standardised variables not actually occurring are irrelevant.
%    \item $\bot$-invariant universal clauses (only having universal variables) are irrelevant.
%    \item invariant under variable elimination removing (existential) variables from
%the clauses does not affect autarkies of A which do not use these variables.
%    \item invariant under renaming renaming variables (existential or universal) is respected by the autarky system.
%\end{enumerate}
%\end{frame}

\begin{frame}{A- and E-systems}
%Consider a DQCNF $F$ and $k \geq 0$: \newline
%\begin{enumerate}
%    \item An $Ak$- autarky for $F$ is an autarky such that all boolean functions assigned depend essentially on at most $k$ variables.
%    \item An $Ek$-autarky is an autarky assigns at most $k$ (existential) variables. 
%\end{enumerate}

\vspace{0.4cm}

  $A1$ allow the boolean functions to  
  depend on 1 universal variable. \newline 

  $E1$ only uses one existential variable. \newline 
  
  We have considered three Autarky Systems \newline 
  \textbf{$A1$, $E1$, $A1 + E1$}.

\end{frame}

%\begin{frame}{Autarky reduction}
% \begin{exampleblock}{Example}
%    {
%  $ F := (\{3,4\}, \{1,2\}, \{\{-1,2\},\{2,-3,-4\}\}, \{(1,\{3\}), (2,\{3,4\})\})$    
%    }
%    \end{exampleblock}
%
%\begin{exampleblock}{A DQDIMACS code is}
%    {
%p cnf 4 2 \\
%a 3 4 0 \\
%e 2 0 \\
%d 1 3 0 \\
%-1 2 0 \\
%2 -3 -4 0 \\       
%    }
%    \end{exampleblock}
%    \vspace{0.2cm}
%\end{frame}

\begin{frame}{Translation to SAT: Finding A1 via compilation}
\begin{enumerate}
    \item \textbf{Selected (boolean) functions}: restrict to some “interesting” selection \textbf{s(v)} $\subseteq$ $BF^{D^v}$ , and treat the elements of s(v) as the possible values of v. \newline 
    
    \item \textbf{Admissible partial assignment:} For each $C \in F$ , compile all possibilities for admissible partial satisfying assignments for C into some CNF-representation \textbf{P(C)}.
\end{enumerate}

\end{frame}

%\begin{frame}{Selected boolean functions}
%Translation \textbf{t} chooses for each $v \in E$ and $f \in s(v)$ a distinct new \textbf{bf-variable} \newline 
%  \textbf{t(v,f)} $\in$  \textbf{Va}   (``true'' iff $f$ is chosen for $v$)
%  
%\begin{equation}
%  \bigwedge_{f, f' \in s(v), f \ne f'} \overline{t(v,f)} \,\, \lor \,\, \overline{t(v,f')}.
% \end{equation}
% 
%\begin{exampleblock}{$s(v)$ all at-most-one-variable-bfs}
%e 2 0 \\
%d 1 3 0 \newline 
%
% % $\abs{s(1)} = 4$ and $\abs{s(2)} = 6$; to list these $4+6 = 10$
%    t(1,bf(0)), t(1,bf(1)), t(1,3), t(1,-3) \newline 
%
%    t(2,bf(0)), t(2,bf(1)), t(2,3), t(2,-3), t(2,4), t(2,-4)
% 
% \end{exampleblock}
%\end{frame}

\begin{frame}{Minimal possibilities for $C$ to become a tautology}
    %For the minimal satisfying clause-assignments we have 
    \begin{enumerate}
        \item some existential $y \in C$ is set to 1;
        \item some existential $y \in C$ there is a universal $x \in C$ with $var(x) \in  D(var(y))$, and y is set to x;
        \item for some existential $y, y' \in C$, $y, y'$ , there is a universal $x \in D(var(y))$, and $y$ is set to $x$ and $y'$ to $x$. 
    \end{enumerate}

\begin{exampleblock}{The admissible autarkies}

$...\exists y_2(x_2,x_3) ... (\ov{y_2} \vee \ov{x_2} \vee x_3)$ \newline
 
$S(\ov{y_2} \vee \ov{x_2} \vee x_3) = \{ \ev{y_2 \to 0}, \ev{y_2 \to \ov{x_2}}, \ev{y_2 \to x_3} \} $ 
\end{exampleblock}

\end{frame}
%
%\begin{frame}{Frame Title}
%We choose distinct new pa-variables
%\[ t(\phi) \in VA 
%\]
%for $\phi \in \bigcup\limits_{C \in F} S(C)$ (`` true" iff $\phi$ is activated).
%
%\begin{equation}
%t(\phi) \iff t(v, \phi(v))
%\end{equation}
%\end{frame}
%
%
%\begin{frame}{Frame Title}
%We choose furthermore distinct new clause-selector-variables
%\[ t(C) \in VA \]
%(``true” if C is satisfied)
%
%for each $C \in F$ the clauses
%\begin{equation}
%    \overline{t(C)} \vee \bigvee_{\phi \in S(C)} t(\phi)
%\end{equation}
%
%\begin{equation}
%    \bigwedge_{v \in var(C) \cap E} \bigwedge_{f \in s(v)} t(C) \bigvee \overline{t(v,f)}
%\end{equation}
%\end{frame}
%
%\begin{frame}{Frame Title}
%    The existence of a non-trivial autarky 
%  \begin{equation}
%    \bigvee_{C \in F} t(C)
%  \end{equation}
%\end{frame}

\begin{frame}{Results}
334 instances in the DQBF track of QBFEVAL'18 

\begin{itemize}
    \item 330 instances are E1+A1-lean (have no non-trivial E1- or A1-autarky). 
    \item From the remaining 4 instances, all are E1-lean, one is A1-satisfiable, 
    \item one is E1+A1-satisfiable, 
    \item the remaining two instances are not E1+A1-satisfiable,
    \item but allow A1-autarkies eliminating in both cases around 80 of variables
and clauses (while adding E1 does not change this).
\end{itemize}

\end{frame}

%\begin{frame}{Results}
%\begin{table}[t]
%  \centering
%\adjustbox{max height=\dimexpr\textheight-5.5cm\relax,
%           max width=\textwidth}{
%\begin{tabular}[t]{|c|c|c|c|c|}\hline
%  
%   {\multirow{2}{*} {\textbf{Solver}}}  & {\multirow{2}{*} {\textbf{Solved instance}}} &  \multicolumn{2}{c|}{\textbf{Instance}} & {\multirow{2}{*} {\textbf{Time}}} \\\cline{3-4}
%  
%   %{} & \multicolumn{2}{c|}{\textbf{2-connected} } & \multicolumn{2}{c|}{\textbf{3-connected} } & \multicolumn{2}{c|}
%   
%  % \\\cline{2-5}
%    {} & {} & {\textbf{Unsat}} & {\textbf{Sat}} & {} \\\hline
%     Rev-Qfun & 220 & 145 & 75 & 572K \\\hline 
%     GhostQ & 194 & 120 & 74 & 617K \\\hline 
%     CAQE & 170 & 128 & 42 & 656K \\\hline 
%     \cellcolor{blue!25}\textsc{RAReQS} & {\color{blue}{\textbf{167}}} & \textbf{133} & 34 & 660K \\\hline 
%     DepQBF & 167 & 121 & 46 & 666K \\\hline 
%    \cellcolor{red!25}\textsc{Heretic} & {\color{red}{163}} &\textbf{ 133} & 30 & 664K \\\hline 
%    \cellcolor{green!25} \textsc{Ijtihad} & {\color{britishracinggreen}{150}} & 127 & 23 & 684K \\\hline 
%%Qute 130 91 39 720K
%%QESTO 109 86 23 761K
%%DynQBF 72 38 34 826K
%%QSTS 152 116 36 687K   
%  \end{tabular}}
% \caption{Original Instances (time include timeouts).}
%  \label{tab:qf-grabh}
%\end{table}
%\end{frame}

\begin{frame}{Conclusion}
\begin{itemize}
\item Autarky theory for DQBF.
\item Three Autarky systems A1, E1, A1+E1
\item A SAT translation 
\end{itemize}
\vspace{1cm}
\Large
Dankeschön!
\end{frame}

\end{document}
