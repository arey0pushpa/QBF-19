%%%%%%%%%%%%%%%%%%%%%%%%%%%%%%%%%%%%%%%%%%%%%%
% Head matter - can we try to be consistent on
% included packages
\documentclass[xcolor=table	]{beamer}
\mode<presentation>
{\usetheme{default}
 \usecolortheme{default}
 \usefonttheme{default}
 \setbeamertemplate{navigation symbols}{}
 \setbeamertemplate{caption}[numbered]} 

\usepackage{syntax}
\usetheme[secheader]{Boadilla}

% Add Better I for interpretation 
\usepackage{mathrsfs}
\usepackage{mathtools}
\DeclarePairedDelimiter\ev{\langle}{\rangle}%
%\usepackage[dvipsnames]{xcolor}
\usepackage[]{ulem}
\usepackage{cancel}
\usepackage{soul}


% color the cell of a table
\usepackage[table]{}% http://ctan.org/pkg/xcolor
\usepackage{multirow}
% Table required packages
\usepackage{hhline}
\usepackage{adjustbox}

\usepackage{array}
%\newtheorem{lemma}{Lemma}
\usepackage[customcolors]{hf-tikz}
\usepackage{tikz}
 \usetikzlibrary{decorations.pathreplacing}

\newcommand{\ov}[1]{\mkern 1.5mu\overline{\mkern-1.5mu#1\mkern-1.5mu}\mkern 1.5mu}


\usepackage{graphicx}
\usepackage{subcaption}
\usepackage[usestackEOL]{stackengine}

\usepackage{xcolor}% http://ctan.org/pkg/xcol
\definecolor{OliveGreen}{rgb}{0.2,0.7,0.2}
\definecolor{airforceblue}{rgb}{0.36, 0.54, 0.66}
\definecolor{awesome}{rgb}{1.0, 0.13, 0.32}
\definecolor{brilliantrose}{rgb}{1.0, 0.33, 0.64}
\definecolor{charcoal}{rgb}{0.21, 0.27, 0.31}
\definecolor{britishracinggreen}{rgb}{0.0, 0.26, 0.15}
\definecolor{darkgreen}{rgb}{0.0, 0.2, 0.13}
\definecolor{darkpastelgreen}{rgb}{0.01, 0.75, 0.24}
\definecolor{lavenderindigo}{rgb}{0.58, 0.34, 0.92}
\definecolor{ultramarine}{RGB}{0,32,96}
\definecolor{mycolor}{rgb}{.5,.05,.05}
\definecolor{britishracinggreen}{rgb}{0.0, 0.26, 0.15}
\definecolor{burntorange}{rgb}{0.8, 0.33, 0.0}
\definecolor{coolblack}{rgb}{0.0, 0.18, 0.39}
\definecolor{harvestgold}{rgb}{0.85, 0.57, 0.0}
\definecolor{harvardcrimson}{rgb}{0.79, 0.0, 0.09}

\definecolor{wrongultramarine}{rgb}{0.07, 0.04, 0.56}

 \setbeamertemplate{footline}{%
   \raisebox{5pt}{\makebox[\paperwidth]{\hfill\makebox[10pt]{\scriptsize\insertframenumber}}}}
%%%%%%%%%%%%%%%%%%%%%%%%%%%%%%%%%%%%%%%%%%%%%%
% Formatting for title page
\title[QBF]{Autarkies for DQCNF}
\author{Oliver Kullmann \inst{1} \and \textbf{Ankit Shukla} \inst{2}}
\institute{\inst{1} Swansea University \inst{2} JKU, Linz}
%{\large International Workshop on QBFs and Beyond}
%\date{10 September 18}
%%%%%%%%%%%%%%%%%%%%%%%%%%%%%%%%%%%%%%%%%%%%%%

\begin{document}
\begin{frame}
  \titlepage
\end{frame}

%\begin{frame}{TODO}
% Later then one defines "autarkies" in general,
% 
% and E1,A1 etc.
% 
% Plus the main lemma, 
% 
% the SAT-equivalence, 
% 
% and explaining "autarky systems" and 
% 
% confluence.
%\end{frame}

%\begin{frame}{DQCNF}
%Dependency Quantified Boolean Formulas (DQBF) extend QBF \\
%with \textbf{Henkin quantifiers}, which allow for non-linear dependencies \\ between the quantified variables. 
%\hfsetfillcolor{green!50}
%\hfsetbordercolor{green!50!black}
%\begin{equation}\label{e:barwq3}
%  \begin{split}
%	\tikzmarkin{c}(0.05,-0.6)(-0.05,0.65)F \, = \, \forall u1, u2 \, \exists e1(u1), e2(u1, u2), e3(u1, u2) . F_0\\
%	F_0 \, = \,(u1 \lor e1) \land (u2 \lor e2) \land (u1 \lor u2 \lor e3)
%	\tikzmarkend{c}
%\end{split}
%\end{equation}	
%\begin{equation}
%\begin{split}
% \overbrace{\tikz[baseline]{
%		\node[fill=yellow!30,anchor=base] (t1)
%		{$\forall x_1, x_2 \, \exists y_1(x_1), y_2(x_1, x_2)$};
%}}^{\displaystyle Dep. \, quantifier \, prefix} : \underbrace{\tikz[baseline]{
%		\node[fill=green!30,anchor=base] (t1)
%		{$(x_1 \lor y_1) \land (x_2 \lor y_2 \lor \neg y_1)$};
%}}^{\displaystyle CNF \, formula}
%\end{split}
%\end{equation}
%
%
%\begin{alertblock}{Dependency set}
% $y_1 \to \{x_1 \}$
%
% $y_2 \to  \{ x_1, x_2 \}$
%\end{alertblock}
%  
%\end{frame}
\strutlongstacks{T}

\begin{frame}{Solve DQCNF: SAT or UNSAT?}

%\vspace{0.2cm}

	\begin{alertblock}{DQCNF formula}
	{		
		\only<1> {
			$F$ :=	$ \forall x, y \, \exists a(x) \, \exists  b(y) \, \exists c(x,y) \, \exists d(x) \, \exists e(y) :
			(a \lor b \lor x) \land  (\ov{a} \lor \ov{b} \lor \ov{x}) \land  (\ov{a} \lor b \lor y) \land  (a \lor \ov{b} \lor \ov{y}) \land \newline 
			(c \lor x \lor y) \land  (c \lor \ov{x} \lor \ov{y} \lor a) \land  (\ov{c} \lor x \lor \ov{y}) \land  (\ov{c} \lor \ov{x} \lor y) \land  \newline 
			(d \lor e \lor x) \land  (\ov{d} \lor \ov{e} \lor \ov{x} \lor c) \land  (\ov{d} \lor e \lor y) \land  (d \lor \ov{e} \lor \ov{y}) $
		}
		\only<2,5-9> {
		$F$ :=	$ \forall x, y \, \exists \textcolor{orange}{a(x)} \, \exists  \textcolor{orange}{b(y)} \, \exists \textcolor{orange}{c(x,y)} \, \exists \textcolor{orange}{d(x)} \, \exists \textcolor{orange}{e(y)} :
		\{a,b,x\}, \{\ov{a},\ov{b},\ov{x}\}, \{\ov{a},b,y\}, \{a,\ov{b},\ov{y}\},  \newline 
		\{c,x,y\}, \{c,\ov{x},\ov{y}, a\}, \{\ov{c},x,\ov{y}\}, \{\ov{c},\ov{x},y\}, \newline 
		\{d,e,x\}, \{\ov{d},\ov{e},\ov{x}, c\}, \{\ov{d},e,y\}, \{d,\ov{e},\ov{y}\} $
	}
  \only<3> {
	$F$ :=	$ \forall x, y \, \exists a(x) \, \exists  b(y) \, \exists c(x,y) \, \exists d(x) \, \exists e(y) :
	\textcolor{blue}{\{a,b,x\}, \{\ov{a},\ov{b},\ov{x}\}, \{\ov{a},b,y\}, \{a,\ov{b},\ov{y}\}},  \newline 
	\{c,x,y\}, \{c,\ov{x},\ov{y}, a\}, \{\ov{c},x,\ov{y}\}, \{\ov{c},\ov{x},y\}, \newline 
	\textcolor{blue}{\{d,e,x\}, \{\ov{d},\ov{e},\ov{x}, c\}, \{\ov{d},e,y\}, \{d,\ov{e},\ov{y}\}} $
  }
  \only<4> {
	F :=	$ \forall x, y \, \exists a(x) \, \exists  b(y) \, \exists c(x,y) \, \exists d(x) \, \exists e(y) :
	\textcolor{blue}{\{a,b,x\}, \{\ov{a},\ov{b},\ov{x}\}, \{\ov{a},b,y\}, \{a,\ov{b},\ov{y}\}},  \newline 
	\{c,x,y\}, \{c,\ov{x},\ov{y}, a\}, \{\ov{c},x,\ov{y}\}, \{\ov{c},\ov{x},y\}, \newline 
	\textcolor{blue}{\{d,e,x\}, \{\ov{d},\ov{e},\ov{x},\textcircled{c}\}, \{\ov{d},e,y\}, \{d,\ov{e},\ov{y}\}} $
}
	}
\end{alertblock}
\pause \pause \pause \pause 

Rules of the game \\

\begin{itemize}
	\pause 
	
	\item Substitute boolean functions of universal variable in place of existential variables.
	\pause
	
	\item The corresponding clauses becomes tautology.
\end{itemize}
\pause
 
Ques. What are allowed values? \\
\pause

Ans. Any boolean function allowed by \textbf{dependency set variables}.
\end{frame}

%\begin{frame}{Solve DQCNF: SAT or UNSAT?}
%
%%\vspace{0.2cm}
%
%	\begin{alertblock}{DQCNF formula}{
%\only<1> {
%		$F$ :=	$ \forall x, y \, \exists a(x) \, \exists  b(y) \, \exists c(x,y) \, \exists d(x) \, \exists e(y) :
%		\{a,b,x\}, \{\ov{a},\ov{b},\ov{x}\}, \{\ov{a},b,y\}, \{a,\ov{b},\ov{y}\}  \newline 
%		\{c,x,y\}, \{c,\ov{x},\ov{y}, a\}, \{\ov{c},x,\ov{y}\}, \{\ov{c},\ov{x},y\} \newline 
%		\{d,e,x\}, \{\ov{d},\ov{e},\ov{x}, c\}, \{\ov{d},e,y\}, \{d,\ov{e},\ov{y}\} $
%	}


% \only<4> {
%	F :=	$ \forall x, y \, \exists a(x) \, \exists  b(y) \, \exists c(x,y) \, \exists d(x) \, \exists e(y) :
%	\textcolor{blue}{\{a,b,x\}, \{\ov{a},\ov{b},\ov{x}\}, \{\ov{a},b,y\}, \{a,\ov{b},\ov{y}\}}  \newline 
%	\{c,x,y\}, \textcolor{red}{\{c,\ov{x},\ov{y}, a\}}, \{\ov{c},x,\ov{y}\}, \{\ov{c},\ov{x},y\} \newline 
%	\textcolor{blue}{\{d,e,x\}, \{\ov{d},\ov{e},\ov{x}, c\}, \{\ov{d},e,y\}, \{d,\ov{e},\ov{y}\}} $
% }
%}
%\end{alertblock}

%\begin{block}{Selected boolean functions}
% \only<2> {
%$a \to 0$, $ a \to 1 \newline$
%$b \to 0$, $ b \to 1 \newline$
%$c \to 0$, $ c \to 1 \newline$
%$d \to 0$, $ d \to 1 \newline$
%$e \to 0$, $ e \to 1 \newline$
%}
% \only<3> {
%$a \to 0$, $ a \to 1, a \to x, a \to \neg x \newline$
%$b \to 0$, $ b \to 1 \newline$
%$c \to 0$, $ c \to 1 \newline$
%$d \to 0$, $ d \to 1 \newline$
%$e \to 0$, $ e \to 1 \newline$
%}
%
%\only<4> {
%	$y_1 \to 0$, $ y_1 \to 1$, $y_1 \to x_1$, $ y_1 \to \ov{x_1}$, $y_1 \to x_2$, $ y_3 \to \ov{x_2} \newline$
%	$y_2 \to 0$, $ y_2 \to 1 \newline$
%	$y_3 \to 0$, $ y_3 \to 1$, $y_3 \to x_1$, $ y_3 \to \ov{x_1}$ 
%}
%
%\only<5> {
%	$y_1 \to 0$, $ y_1 \to 1$, $y_1 \to x_1$, $ y_1 \to \ov{x_1}$, $y_1 \to x_2$, $ y_1 \to \ov{x_2} \newline$
%	$y_2 \to 0$, $ y_2 \to 1, y_2 \to x_2$, $ y_2 \to \ov{x_2}$, $y_2 \to x_3$, $ y_2 \to \ov{x_3} \newline$
%	$y_3 \to 0$, $ y_3 \to 1$, $y_3 \to x_1$, $ y_3 \to \ov{x_1}$ 
%}
%\end{block}

%\end{frame}


\begin{frame}{Solve DQCNF piece by piece}
\begin{alertblock}{DQCNF formula} {
\only<1>	{	
$F$ :=	$ \forall x, y \, \exists a(x) \, \exists  b(y) \, \exists c(x,y) \, \exists d(x) \, \exists e(y) :
	\{a,b,x\}, \{\ov{a},\ov{b},\ov{x}\}, \{\ov{a},b,y\}, \{a,\ov{b},\ov{y}\},  \newline 
	\{c,x,y\}, \{c,\ov{x},\ov{y}, a\}, \{\ov{c},x,\ov{y}\}, \{\ov{c},\ov{x},y\}, \newline 
	\{d,e,x\}, \{\ov{d},\ov{e},\ov{x}, c\}, \{\ov{d},e,y\}, \{d,\ov{e},\ov{y}\} $
}
\only<2-4>{
$F$ :=	$ \forall x, y \, \exists a(x) \, \exists  b(y) \, \exists c(x,y) \, \exists d(x) \, \exists e(y) :
    \{a,b,x\}, \{\ov{a},\ov{b},\ov{x}\}, \{\ov{a},b,y\}, \{a,\ov{b},\ov{y}\},  \newline 
	\{c,x,y\}, \{c,\ov{x},\ov{y}, a\}, \{\ov{c},x,\ov{y}\}, \{\ov{c},\ov{x},y\}, \newline 
		\textcolor{blue}{\{d,e,x\}, \{\ov{d},\ov{e},\ov{x}, c\}, \{\ov{d},e,y\}, \{d,\ov{e},\ov{y}\}} $
}

\only<5>{$F_1$ :=	$ \forall x, y \,  \exists a(x) \, \exists  b(y) \, \exists c(x,y) : \newline 
\{a,b,x\}, \{\ov{a},\ov{b},\ov{x}\}, \{\ov{a},b,y\}, \{a,\ov{b},\ov{y}\},  \newline 
	\{c,x,y\}, \{c,\ov{x},\ov{y}, a\}, \{\ov{c},x,\ov{y}\}, \{\ov{c},\ov{x},y\}, \newline 
	\textcolor{blue}{	\{\ov{x}, \ov{y},x\}, \{x,y,\ov{x}, c\}, \{x,\ov{y},y\}, \{\ov{x},y,\ov{y}\}} $ }  
%    \pause 

\only<6>{ $F_1$ :=	$ \forall x, y \,  \exists a(x) \, \exists  b(y) \, \exists c(x,y) : \newline 
\{a,b,x\}, \{\ov{a},\ov{b},\ov{x}\}, \{\ov{a},b,y\}, \{a,\ov{b},\ov{y}\},  \newline 
	\{c,x,y\}, \{c,\ov{x},\ov{y}, a\}, \{\ov{c},x,\ov{y}\}, \{\ov{c},\ov{x},y\}, \newline 
	\textcolor{blue}{	\cancel{\{\ov{x},e,x\}}, \cancel{\{x,y,\ov{x}, c\}}, \cancel{\{x,\ov{y},y\}}, \cancel{\{\ov{x},y,\ov{y}\}}} $ }  
%    \pause 

\only<7>{ $F_1$ :=	$ \forall x, y \,  \exists a(x) \, \exists  b(y) \, \exists c(x,y) : \newline 
\{a,b,x\}, \{\ov{a},\ov{b},\ov{x}\}, \{\ov{a},b,y\}, \{a,\ov{b},\ov{y}\},  \newline  
	\{c,x,y\}, \{c,\ov{x},\ov{y}, a\}, \{\ov{c},x,\ov{y}\}, \{\ov{c},\ov{x},y\}$ } 

}
\end{alertblock}

\begin{exampleblock}{Solve the DQCNF}
	{   \only<1-2> {Choices $a, b, c, d, e$ \newline} 
		\pause \pause
	\only<3->	{\begin{itemize}
		%	\pause
		\only \item Pick $d, e$\\
		\pause 
		  $d \to \neg x$, $e \to \neg y$ \newline 
		\pause   
		%	Both these autarkies are also $Eaut_1$-autarkies.
		\end{itemize}
	}
	}
\end{exampleblock}

%\textbf{lean kernel} of the original formula.
\end{frame}

\begin{frame}{Solve DQCNF piece by piece}
\begin{alertblock}{DQCNF formula} {
		\only<1>	{	
			$F_1$ :=	$ \forall x, y \,  \exists a(x) \, \exists  b(y) \, \exists c(x,y) : \newline 
				\{a,b,x\}, \{\ov{a},\ov{b},\ov{x}\}, \{\ov{a},b,y\}, \{a,\ov{b},\ov{y}\},  \newline 
			\{c,x,y\}, \{c,\ov{x},\ov{y}, a\}, \{\ov{c},x,\ov{y}\}, \{\ov{c},\ov{x},y\} $
		}
		\only<2>	{	
		$F_1$ :=	$ \forall x, y \,  \exists a(x) \, \exists  b(y) \, \exists c(x,y) : \newline 
		\textcolor{blue}{\{a,b,x\}, \{\ov{a},\ov{b},\ov{x}\}, \{\ov{a},b,y\}, \{a,\ov{b},\ov{y}\}},  \newline 
		\{c,x,y\},\{c,\ov{x},\ov{y}, a\}, \{\ov{c},x,\ov{y}\}, \{\ov{c},\ov{x},y\} $
	}
	\only<3>	{	
	$F_1$ :=	$ \forall x, y \,  \exists a(x) \, \exists  b(y) \, \exists c(x,y) : \newline 
	\textcolor{blue}{\{a,b,x\}, \{\ov{a},\ov{b},\ov{x}\}, \{\ov{a},b,y\}, \{a,\ov{b},\ov{y}\}},  \newline 
	\{c,x,y\},  \textcolor{red}{\{c,\ov{x},\ov{y}, a\}}, \{\ov{c},x,\ov{y}\}, \{\ov{c},\ov{x},y\} $
}
		\only<4-6>{
		$F_1$ :=	$ \forall x, y \,  \exists a(x) \, \exists  b(y) \,\exists c(x,y) : \newline 
			\{a,b,x\}, \{\ov{a},\ov{b},\ov{x}\}, \{\ov{a},b,y\}, \{a,\ov{b},\ov{y}\},  \newline 
	\textcolor{blue} {\{c,x,y\}, \{c,\ov{x},\ov{y}, a\}, \{\ov{c},x,\ov{y}\}, \{\ov{c},\ov{x},y\}} $
		}
	
		\only<7>{ $F_2$ :=	$ \forall x, y \,  \exists a(x) \, \exists  b(y)  : \newline 
		\{a,b,x\}, \{\ov{a},\ov{b},\ov{x}\}, \{\ov{a},b,y\}, \{a,\ov{b},\ov{y}\},  \newline 
		\textcolor{blue} {\{ x \Leftrightarrow y,x,y\}, \{ x \Leftrightarrow y,\ov{x},\ov{y}, a\}, \{\ov{x \Leftrightarrow y},x,\ov{y}\}, \{\ov{x \Leftrightarrow y},\ov{x},y\}}  $ }  
	%    \pause 
	
	\only<8>{ $F_2$ :=	$ \forall x, y \,  \exists a(x) \, \exists  b(y)  : \newline 
	\{a,b,x\}, \{\ov{a},\ov{b},\ov{x}\}, \{\ov{a},b,y\}, \{a,\ov{b},\ov{y}\},  \newline 
	\textcolor{blue} { \cancel{\{ x \Leftrightarrow y,x,y\}},  \cancel{\{ x \Leftrightarrow y,\ov{x},\ov{y}, a\}},  \cancel{\{\ov{x \Leftrightarrow y},x,\ov{y}\}},  \cancel{\{\ov{x \Leftrightarrow y}}, \cancel{\ov{x},y\}}}  $ }  
	%    \pause  
	
	\only<9>{ $F_2$ :=	$ \forall x, y \,  \exists a(x) \, \exists  b(y)  : \newline 
		\{a,b,x\}, \{\ov{a},\ov{b},\ov{x}\}, \{\ov{a},b,y\}, \{a,\ov{b},\ov{y}\}
		$ }
	}
\end{alertblock}

\begin{exampleblock}{Solve the DQCNF}
	{   \pause \pause \pause \pause
		\begin{itemize}
			%	\pause
			\only \item Pick $c$\\
			\pause 
			$c = x \Leftrightarrow y  $
			%\,\,\,\,\, (x \lor \neg y) \land (\neg x \lor  y)$ 
			\newline   
		%    \pause 
		\end{itemize}
	}
\end{exampleblock}

%\textbf{lean kernel} of the original formula.
\end{frame}

\begin{frame}{Solve DQCNF piece by piece}
\begin{alertblock}{DQCNF formula} {
		\only<1>	{	
			$F_2$ :=	$ \forall x, y \,  \exists a(x) \, \exists  b(y) : \newline 
			\{a,b,x\}, \{\ov{a},\ov{b},\ov{x}\}, \{\ov{a},b,y\}, \{a,\ov{b},\ov{y}\} $
		}
		\only<2-4>{
				$F_2$ :=	$ \forall x, y \,  \exists a(x) \, \exists  b(y) : \newline 
		\textcolor{blue}{	\{a,b,x\}, \{\ov{a},\ov{b},\ov{x}\}, \{\ov{a},b,y\}, \{a,\ov{b},\ov{y}\} }$
		}
		\only<5>{ $F_3$ :=	 $
		\textcolor{blue}{	\{\ov{x}, \ov{y},x\}, \{x, y,\ov{x}\}, \{x,\ov{y},y\}, \{\ov{x},y,\ov{y}\}}
		$ }  
	%    \pause 
	
	\only<6>{ $F_3$ :=	$\textcolor{blue}{	\cancel{\{\ov{x}, \ov{y},x\}}, \cancel{\{x, y,\ov{x}\}}, \cancel{\{x,\ov{y},y\}}, \cancel{\{\ov{x},y,\ov{y}\}}}		$  }
	%    \pause  
	
	\only<7->{ $F_3$ :=	true }  
	}
\end{alertblock}

\begin{exampleblock}{Solve the DQCNF}
	{   \pause \pause 
		\begin{itemize}
			%	\pause
			\only \item Pick $a, b$\\
			\pause 
			$a = \neg x, b = \neg y$ \newline 
			%    \pause 
		\end{itemize}
	}


\end{exampleblock}
\end{frame}

\begin{frame}{DQCNF F is SAT}
\begin{alertblock}{DQCNF formula} {
			$F$ :=	$ \forall x, y \, \exists a(x) \, \exists  b(y) \, \exists c(x,y) \, \exists d(x) \, \exists e(y) :
			\{a,b,x\}, \{\ov{a},\ov{b},\ov{x}\}, \{\ov{a},b,y\}, \{a,\ov{b},\ov{y}\},  \newline 
			\{c,x,y\}, \{c,\ov{x},\ov{y}, a\}, \{\ov{c},x,\ov{y}\}, \{\ov{c},\ov{x},y\}, \newline 
			\{d,e,x\}, \{\ov{d},\ov{e},\ov{x}, c\}, \{\ov{d},e,y\}, \{d,\ov{e},\ov{y}\} $
		}
\end{alertblock}
\pause 
	\begin{exampleblock}{Total satisfying assignment} {
			$ a \to \neg x, \, b \to \neg y, \, c \to (x \Leftrightarrow  y), \, d \to \neg x, \, e \to \neg y \newline $
		}
    \end{exampleblock} 
%\textbf{lean kernel} of the original formula.
\end{frame}

%\begin{frame}{Autarkies for DQBF}
%``Autarkies" are some form of redundancies in CNFs, \newline
%with good theoretical properties \newline
%
%We present Autarky-system to determine these redundancies for DQBF \newline 
%
%%A1, E1, A1 + E1 \newline
%
%Present complete translation using an example 
%
%\end{frame}

%\begin{frame}{Autarkies}
%
%Autarkies for SAT are partial assignments for boolean CNF, \newline 
% which either satisfy a clause or leave it untouched. \newline
% 
%$F$ =  $(a) \land (a \lor b)$ \newline
% 
%Nontrivial Autarky:  $<b \to 1>$ \newline 
%
%Reduced formula: \newline 
%$F^{\prime}$ =  $(a)$
% 
%\end{frame}

\begin{frame}{Autarkies for DQCNF}
%\textit{An autarky for a clause-set $F$ is 
\begin{block}{Autarkies for SAT}
	A \textbf{partial assignment} $\varphi: var(F) \to \{0,1\}$ is an autarky iff 
	\begin{itemize}
		\item for every clause $C \in F$ either $\varphi$ does not ``touch" $C$, i.e., $var(\varphi) \cap var(C) = \emptyset$, or
		\item  $\varphi$ satisfies $C$ i.e. $\varphi \ast C$ is true. 
	\end{itemize}
\pause 
\end{block}
\begin{exampleblock}{Example}
	For $F$ = $\{a\}, \textcolor{blue}{\{a, b\}}$; \pause the partial assignment $b \to 1$ is an autarky. 
\end{exampleblock}
%	\newline

%	$\overbrace{\tikz[baseline]{
%			\node[fill=yellow!50,anchor=base] (t1)
%			{\varphi \ast \{C\} = \top};} 
%	\rule[-12pt]{0pt}{5pt}}^{\mbox{Application of partial assignmt}} $
%\begin{equation}
%\begin{split} 
%& \underbrace{\tikz[baseline]{%
%		\node[fill=blue!30,anchor=base] (t1)
%		{$\varphi \ast \{C\} = \top$};
%	} \rule[-12pt]{0pt}{5pt}}_{\mbox{$A$}} \bar{w}_p                  
%\end{split}                     
%\end{equation}

\pause 

 \begin{alertblock}{Autarkies for DQCNF}
	\begin{itemize}
	 \item $\varphi$
	%	maps each existentially quantified variable of F to a Boolean function,	
	assign existential variables v of F with \textbf{boolean functions} as allowed by the dependency-set.
	
	\item Making a clause ``true" now means making it a tautology.
	
	\end{itemize}
	
\end{alertblock}
%\begin{itemize}
%	 \item $\varphi$
%%	maps each existentially quantified variable of F to a Boolean function,	
%	assign existential variables v of F, with values now being \newline \textbf{boolean functions} as allowed by the dependency-set. \newline
%	
%	\item Making a clause ``true" now means making it a tautology.
%\end{itemize}

%\begin{figure}[h!]
%	\centering
%	\begin{subfigure}[b]{0.4\linewidth}
	%	\includegraphics[width=\linewidth]{zw}
%		\caption{Coffee.}
%	\end{subfigure}
%	\begin{subfigure}[b]{0.4\linewidth}
%		\includegraphics[width=\linewidth]{coffee.jpg}
%		\caption{More coffee.}
%	\end{subfigure}
%	\caption{The same cup of coffee. Two times.}
%	\label{fig:coffee}
%\end{figure}
%
%%($var(\phi) \, \cap \, var(C)$ $\neq$ $\phi$)
%An autarky $\phi$ for $F$ is \textbf{trivial} if $var(\phi) \, \cap \, var(C)$ = $\phi$ \newline
\end{frame}


\begin{frame}{DQCNF Autarky}
\begin{exampleblock}{DQCNF formula} {
	
		\only{
			$F$ :=	$ \forall x, y \, \exists a(x) \, \exists  b(y) \, \exists c(x,y) \, \exists d(x) \, \exists e(y) :
			\{a,b,x\}, \{\ov{a},\ov{b},\ov{x}\}, \{\ov{a},b,y\}, \{a,\ov{b},\ov{y}\},  \newline 
			\{c,x,y\}, \{c,\ov{x},\ov{y}, a\}, \{\ov{c},x,\ov{y}\}, \{\ov{c},\ov{x},y\}, \newline 
			\textcolor{blue}{\{d,e,x\}, \{\ov{d},\ov{e},\ov{x}, c\}, \{\ov{d},e,y\}, \{d,\ov{e},\ov{y}\}} \newline $
		}
		
	 \pause
		Partial assignment, \newline 	$d \to \neg x$, $e \to \neg y$ is an autarky.
			%	Both these autarkies are also $Eaut_1$-autarkies.
	}
\end{exampleblock}

%\textbf{lean kernel} of the original formula.
\end{frame}

\begin{frame}{The interesting cases}

\begin{enumerate}
	\item The empty partial assignment is an autarky for every F \newline (trivial autarky).\newline 
	
	\item A satisfying assignment for F is also an autarky for F. \newline
	\pause 
	\item Elimination of \textbf{pure literals} is a special case of an ``autarky reduction". 
	\newline
	\begin{exampleblock}{Pure literal}
			$\forall x, y \exists a(x) b(x,y) : \{x, \ov{y}, \ov{a}\}, \textcolor{blue}{\{\ov{x}, a, b\}, \{y, b\}} $ \newline 
		Assign $b \to 1$.
	\end{exampleblock}

	%A literal $x$ is \textit{pure} if $\ov{x}$ does not occur in $F$. \newline 
	%Yields an autarky, $x \to 1$ for $F$.
\end{enumerate}
%\pause 
%\vspace{0.5cm}

%A \textbf{lean} clause-set is a clause-set $F$ which has
%only the trivial autarky. \newline 

%The class of all lean clause-sets is $lean$.

\end{frame}


\begin{frame}{Autarkies use and challenges}

Applications: 
\begin{enumerate}
    \item Study of minimal unsatisfiability~\cite{buning2009minimal}
    \item Pre-processing 
    \item Inprocessing
\end{enumerate}

 \begin{exampleblock}{Challenge}
 Finding an autarky for DQCNF is as hard as finding a satisfying assignment.
\end{exampleblock}
    \pause 
  \begin{alertblock}{Our solution: Autarky Systems}
  \begin{itemize}
      \item restricting the range of autarkies to a more feasible domain.
      \item maintain the good general properties of arbitrary autarkies~\cite{buning2009minimal}.
  \end{itemize}
 
  \end{alertblock}

\end{frame}

\begin{frame}{The basic lemmas}
 \begin{lemma}[satisfiability-equivalence]
  For an autarky $\varphi$ of $F$, $\varphi \ast F$ is satisfiability-equivalent to $F$.
 \end{lemma}
%\pause
\only<1>{\begin{proof}
 	If there is a satisfying assignment of F it satisfies also
 	$\varphi \ast F$, since just clauses have been removed. 
 	
 	If $\phi$ is a total satisfying assignment for $\varphi \ast F$, then $\varphi \cup \phi$ is a (partial) satisfying assignment for F.
 \end{proof}
}
\only<2->{
	\begin{lemma}[confluence]
		Autarky reduction is confluent.  
	\end{lemma}
%}
%\only<3->{
	\begin{lemma}[composition]
		The composition of two autarkies is again an autarky.
	\end{lemma}
}
\end{frame}

\begin{frame}{The basic lemmas}
 \begin{itemize}
 \item A DQCNF F is called \textbf{lean} if it has no non-trivial autarkies. \newline
 	\pause 
 	
 \item	\textbf{Lean kernel} (unique) is obtained by repeatedly applying autarky-reduction on $F$ as long as possible. \newline
 \end{itemize}
  
% \end{lemma}
 \pause 
  \begin{lemma}[autarky decomposition]
  	A DQCNF can always be decomposed into largest
  	autark sub-DQCNF (satisfiable part by autarky) and largest lean sub-DQCNF (lean kernel ). \newline
  	
 	Every chain of autarky reductions starting
 	with F can be extended \newline to it's lean kernel (where
 	it necessarily ends).
 \end{lemma}
\vspace{0.5cm}
 
\end{frame}


%\begin{frame}{Autarky systems}
%
%Allow restricted notions of autarkies. \newline
%\begin{itemize}
%    \item For a DQCNF $F$ we write $Auk(F)$ for the set of all autarkies.  
%    \item An autarky system $A$ allows to consider subsets $A(F) \subseteq Auk(F)$.
%\end{itemize}
%
%\vspace{0.5cm}
%
%Required conditions for an autarky system:
%\begin{enumerate}
%    \item DQCNF $F$ and $\phi$, $\psi \in A(F)$ it
%must always hold $\phi \circ \psi \in A(F)$ (closure under composition).
%
%    \item DQCNF $F \subseteq F_0$ it must always hold $A(F_0) \subseteq A(F)$ (removal of clauses does not remove A-autarkies).
%\end{enumerate}
%
%A-satisfiability means satisfiability by a series of A-
%autarkies. \newline
%
%A-leanness means there there are no nontrivial A-autarkies.
%\end{frame}

%\begin{frame}{Additional concepts}
%
%Fundamental conditions on the Autarky system:
%
%\begin{enumerate}
%    \item standardised variables not actually occurring are irrelevant.
%    \item $\bot$-invariant universal clauses (only having universal variables) are irrelevant.
%    \item invariant under variable elimination removing (existential) variables from
%the clauses does not affect autarkies of A which do not use these variables.
%    \item invariant under renaming renaming variables (existential or universal) is respected by the autarky system.
%\end{enumerate}
%\end{frame}

%\begin{frame}{A- and E-systems}
%%Consider a DQCNF $F$ and $k \geq 0$: \newline
%%\begin{enumerate}
%%    \item An $Ak$- autarky for $F$ is an autarky such that all boolean functions assigned depend essentially on at most $k$ variables.
%%    \item An $Ek$-autarky is an autarky assigns at most $k$ (existential) variables. 
%%\end{enumerate}
%
%\vspace{0.4cm}
%
%We have considered three Autarky Systems \newline 
%\textbf{$A_1$, $E_1$, $E_1 + A_1$}. \newline
%
%\begin{enumerate}
%	\item $A_1$ allow the boolean functions to  
%	depend on 1 \newline universal variable. \newline
%	
%%	Deciding whether a DQCNF has a non-trivial
%%	$A_1$-autarky is NP-complete.
%	
%	\item $E_1$ only uses one existential variable. \newline
%%	Deciding the existence and finding some short $E_1$-autarky can be done in	polynomial time.
%	
%	\item $A_1 + E_1$, combination of $A_1$ and $E_1$-autarky system. 
%\end{enumerate}
%
%\end{frame}

\begin{frame}{A- and E-systems: $A_1$ Autarky system}
%Consider a DQCNF $F$ and $k \geq 0$: \newline
%\begin{enumerate}
%    \item An $Ak$- autarky for $F$ is an autarky such that all boolean functions assigned depend essentially on at most $k$ variables.
%    \item An $Ek$-autarky is an autarky assigns at most $k$ (existential) variables. 
%\end{enumerate}

\vspace{0.2cm}
\only
 $A_1$ allow the boolean functions to  
	depend on 1 universal variable. 
	
	\begin{alertblock}{DQCNF formula}
		{		
		F :=	$ \forall x, y \, \exists a(x) \, \exists  b(y) \, \exists c(x,y) \, \exists d(x) \, \exists e(y) :
			\{a,b,x\}, \{\ov{a},\ov{b},\ov{x}\}, \{\ov{a},b,y\}, \{a,\ov{b},\ov{y}\},  \newline 
			\{c,x,y\}, \{c,\ov{x},\ov{y}, a\}, \{\ov{c},x,\ov{y}\}, \{\ov{c},\ov{x},y\}, \newline 
			\{d,e,x\}, \{\ov{d},\ov{e},\ov{x}, c\} \{\ov{d},e,y\}, \{d,\ov{e},\ov{y}\} $
		}
	\end{alertblock}
    
    \begin{example}
    	Exactly one $A_1$-autarky \newline 
    	 $a \to \neg x$, $b \to \neg y \newline$
    	 $F$ is $E_1$-lean.
    \end{example}

Deciding whether a DQCNF has a non-trivial $A_1$-autarky is \textbf{NP-complete}.
\end{frame}
    
\begin{frame}{A- and E-systems: $E_1$ Autarky system}
$E_1$ only uses one existential variable. 
\begin{alertblock}{DQCNF formula}
	{		
		$F_1$ :=	$ \forall x, y \, \exists c(x,y) \, \exists d(x) \, \exists e(y) : \newline 
		\{c,x,y\}, \{c,\ov{x},\ov{y}, a\}, \{\ov{c},x,\ov{y}\}, \{\ov{c},\ov{x},y\}, \newline 
		\{d,e,x\}, \{\ov{d},\ov{e},\ov{x}, c\} \{\ov{d},e,y\}, \{d,\ov{e},\ov{y}\} $
	}
\end{alertblock}

 \begin{example}
	Exactly one $E_1$-autarky \newline 
	$c \to (x \lor \neg y) \land (\neg x \lor y)$
	
	$F_1$ $A_1$-lean.
	
\end{example} 

Deciding the existence of $E_1$-autarky can be done in
\textbf{polynomial time}.
\end{frame}

%\begin{frame}{Autarky reduction}
% \begin{exampleblock}{Example}
%    {
%  $ F := (\{3,4\}, \{1,2\}, \{\{-1,2\},\{2,-3,-4\}\}, \{(1,\{3\}), (2,\{3,4\})\})$    
%    }
%    \end{exampleblock}
%
%\begin{exampleblock}{A DQDIMACS code is}
%    {
%p cnf 4 2 \\
%a 3 4 0 \\
%e 2 0 \\
%d 1 3 0 \\
%-1 2 0 \\
%2 -3 -4 0 \\       
%    }
%    \end{exampleblock}
%    \vspace{0.2cm}
%\end{frame}

\begin{frame}{Translation (\textbf{t}) to SAT: Finding $A_1$ via compilation}
\begin{enumerate}
    \item \textbf{Selected (boolean) functions}: 
    explicitly list the possible boolean functions as values of the existential variables. \newline
 
    %$\textcolor{orange}{\exists c(x,y)}$ \newline 
    $\texttt{S(c)} = t(c, 0) , t(c, 1), t(c, x), t(c, \neg x), t(c, y), t(c, \neg y)$	 \newline 

   \pause   
    %restrict to some ``interesting" selection \textbf{s(v)} $\subseteq$ boolean function, and treat the elements of s(v) as the possible values of v. \newline 
  
    \item\textbf{Admissible partial assignment}:
    compile for each clause $C \in F$ the minimal possibilities
    for $C$ to become a tautology. \newline 
    
    $\texttt{M} (\{\texttt{c},\ov{\texttt{x}},\ov{\texttt{y}}, \texttt{a}\})$ = 
    $p(c, 1) , p(a, 1),
    \pause 
    p(c, x), p(c, y), p(a, x), \newline 
    \pause 
   \hspace{2.75cm} p(c, x, a, \neg x), p(c, \neg x, a, x) $ \newline 
    %For each $C \in F$ , \textbf{compile} all possibilities for admissible partial satisfying assignments for C into some CNF-representation \textbf{P(C)}.
    \pause
    \item \textbf{Selector-variable}: if $C$ is touched (selected), at least one of the minimal possibilities for $C$ is fulfilled. \newline 
    
    \texttt{ALO}($\texttt{M} (\{\texttt{c},\ov{\texttt{x}},\ov{\texttt{y}} \})$)
\end{enumerate}

\end{frame}

%\begin{frame}{Selected boolean functions}
%Translation \textbf{t} chooses for each $v \in E$ and $f \in s(v)$ a distinct new \textbf{bf-variable} \newline 
%  \textbf{t(v,f)} $\in$  \textbf{Va}   (``true'' iff $f$ is chosen for $v$)
%  
%\begin{equation}
%  \bigwedge_{f, f' \in s(v), f \ne f'} \overline{t(v,f)} \,\, \lor \,\, \overline{t(v,f')}.
% \end{equation}
% 
%\begin{exampleblock}{$s(v)$ all at-most-one-variable-bfs}
%e 2 0 \\
%d 1 3 0 \newline 
%
% % $\abs{s(1)} = 4$ and $\abs{s(2)} = 6$; to list these $4+6 = 10$
%    t(1,bf(0)), t(1,bf(1)), t(1,3), t(1,-3) \newline 
%
%    t(2,bf(0)), t(2,bf(1)), t(2,3), t(2,-3), t(2,4), t(2,-4)
% 
% \end{exampleblock}
%\end{frame}

%\begin{frame}{Minimal possibilities for $C$ to become a tautology}
%%Compile for each clause the minimal possibilities to make it a tautology.
%    \begin{enumerate}
%        \item some existential $y \in C$ is set to 1;
%        \item some existential $y \in C$ there is a universal $x \in C$ with $var(x) \in  D(var(y))$, and y is set to x;
%        \item for some existential $y, y' \in C$, $y, y'$ , there is a universal $x \in D(var(y))$, and $y$ is set to $x$ and $y'$ to $x$. 
%    \end{enumerate}
%
%\begin{exampleblock}{The admissible autarkies}
%
%$...\exists y_2(x_2,x_3) ... (\ov{y_2} \vee \ov{x_2} \vee x_3)$ \newline
% 
%$S(\ov{y_2} \vee \ov{x_2} \vee x_3) = \{ \ev{y_2 \to 0}, \ev{y_2 \to \ov{x_2}}, \ev{y_2 \to x_3} \} $ 
%\end{exampleblock}

%\end{frame}
%
%\begin{frame}{Frame Title}
%We choose distinct new pa-variables
%\[ t(\phi) \in VA 
%\]
%for $\phi \in \bigcup\limits_{C \in F} S(C)$ (`` true" iff $\phi$ is activated).
%
%\begin{equation}
%t(\phi) \iff t(v, \phi(v))
%\end{equation}
%\end{frame}
%
%
%\begin{frame}{Frame Title}
%We choose furthermore distinct new clause-selector-variables
%\[ t(C) \in VA \]
%(``true” if C is satisfied)
%
%for each $C \in F$ the clauses
%\begin{equation}
%    \overline{t(C)} \vee \bigvee_{\phi \in S(C)} t(\phi)
%\end{equation}
%
%\begin{equation}
%    \bigwedge_{v \in var(C) \cap E} \bigwedge_{f \in s(v)} t(C) \bigvee \overline{t(v,f)}
%\end{equation}
%\end{frame}
%
%\begin{frame}{Frame Title}
%    The existence of a non-trivial autarky 
%  \begin{equation}
%    \bigvee_{C \in F} t(C)
%  \end{equation}
%\end{frame}

\begin{frame}{Numbers: Computing normalforms in DQBF track}
334 instances in the DQBF track of QBFEVAL'18 in 9000s. \newline
(finding an autarky is quick, proving UNSAT: time consuming.)\newline

330 instances are $E_1$+$A_1$-lean (have no non-trivial $E_1$- or $A_1$-autarky). 

%\begin{itemize}
%    \item 330 instances are $E_1$+$A_1$-lean (have no non-trivial $E_1$- or $A_1$-autarky). 
%    \item From the remaining 4 instances, all are $E_1$-lean, one is $A_1$-satisfiable, 
%    \item one is $E_1 + A_1$-satisfiable, 
%    \item the remaining two instances are not $E_1 + A_1$-satisfiable,
%    \item but allow $A_1$-autarkies eliminating in both cases around 80 of variables
%and clauses (while adding $E_1$ does not change this).
%\end{itemize}
%\end{frame}
%\begin{frame}{Results}
\begin{table}[t]
  \centering
\adjustbox{max height=\dimexpr\textheight-5.5cm\relax,
           max width=\textwidth}{
\begin{tabular}[t]{|c|c|c|c|c|}\hline
  
{\multirow{2}{*} {\textbf{No.}}} &   {\multirow{2}{*} {\textbf{Instances}}}  & {\multirow{2}{*} {\textbf{Autarky type}}} &  \multicolumn{2}{c|}{\textbf{Reduction}}  \\\cline{4-5}
  
   %{} & \multicolumn{2}{c|}{\textbf{2-connected} } & \multicolumn{2}{c|}{\textbf{3-connected} } & \multicolumn{2}{c|}
   
  % \\\cline{2-5}
   {} & {} & {} & {\textbf{c(F)}} & {\textbf{c(F')}} \\\hline
%    {0.} & 330 instances  & $E_1$+$A_1$-lean & - & - \\\hline 
     {1.} & \cellcolor{blue!18}\textsc{bloem\_eq1.dqdimacs} & {\color{blue}{\textbf{A1-satisfiable}}} & \textbf{-} & - \\\hline 
   {2.} &   \cellcolor{red!25}{ \Centerstack{\textsc{tentrup17\_ltl2dba\_theta}\\\textsc{\_environment\_1.dqdimacs}}} & {\color{red}{E1+A1-satisfiable}} & - & - \\\hline 
    {3.} & \cellcolor{green!25}\textsc{bloem\_ex1.dqdimacs} & {\color{britishracinggreen}{A1: non-trivial autarky}} & 52 & \textbf{18}  \\\hline 
  {4.} &   \cellcolor{green!25} \textsc{bloem\_ex2.dqdimacs} & {\color{britishracinggreen}{A1: non-trivial autarky}} & 139 & 99 \\\hline 
%Qute 130 91 39 720K
%QESTO 109 86 23 761K
%DynQBF 72 38 34 826K
%QSTS 152 116 36 687K   
  \end{tabular}
}
% \caption{Original Instances (time include timeouts).}
  \label{tab:qf-grabh}
\end{table}
\end{frame}

\begin{frame}{Conclusion}
\begin{itemize}
\item Autarky theory for DQBF.
\item Three autarky systems $A_1$, $E_1$, $E_1 + A_1$.
\item A SAT translation. 
\end{itemize}
\vspace{0.5cm}
\textbf{Future Work: \newline}
\begin{itemize}
	\item Determining the (unique) normalforms for
	$A_1$, $E_1$, $E_1$+$A_1$ for all over 12,000 instances in QBFLIB.
	\item Consider more stronger autarky systems $A_2, E_2$.
%	\item Perform inprocessing.
\end{itemize}
\vspace{0.5cm}
\Large
Thanks!
\end{frame}

\begin{frame}{Bibliography}
\bibliographystyle{plainnat}
\bibliography{qbf19}
\end{frame}

\end{document}
